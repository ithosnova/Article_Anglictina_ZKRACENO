
\section{Materials and methods}
\label{s:materials}

    \subsection{Nanoparticle synthesis}
    
        The synthesis of ZnO:Ga was based on previous works \cite{Prochazkova2015,Vanecek2017}. 
        For stock solution of gallium (III) ions \numprint[g]{0.0327} of gallium (III) oxide was dissolved in \numprint[ml]{1} of nitric acid and the solution was topped up to \numprint[ml]{10} by water. 
        \numprint[g]{4.07} of zinc oxide (ZnO) was dissolved in \numprint[ml]{3,9} of formic acid, \numprint[ml]{1} of stock solution and \numprint[ml]{200} of hydrogen peroxide was added and the solution was filled up to \numprint[ml]{2000} by water. Aqueous solution was irradiated by low-pressure mercury lamps for \numprint[min]{100}. 
        Prepared zinc peroxide was separated by filtration, washed by water and EtOH and dried at \numprint[\celsius]{40} in air. The product was recrystallised for \numprint[min]{60} at \numprint[\celsius]{200} and calcined for \numprint[min]{120} at \numprint[\celsius]{1000}. Crystalline ZnO was annealing in reduction atmosphere of~H$_{2}$/Ar mixture (1:20) for \numprint[min]{60} at about \numprint[\celsius]{800}.\\
     
        Surface modification of nanoparticles was based on sol-gel process published in Chemistry of Materials journal \cite{Liu1998} and modified according to Ing.~Vojtěch Vaněček master thesis \cite{Vanecek2017}. Porphyrin nanoconjugates were prepared according to work of Nowostawska et al. \cite{nowostawska11}.

    \subsection{Luminescence and photoluminescence of ZnO:Ga and ZnO:Ga@SiO$_{2}$-PPIX} \label{sec:character_zno}
    
    Radioluminescence spectra were measured with \mbox{X-ray} source on FZU, \numprint[kV]{40} and \numprint[mA]{15}. Approximately \numprint[mg]{15} were dispersed in plastic cuvette with EtOH and the cuvette was placed into the device.
    
    Decay curves and PL emission spectra of ZnO:Ga@SiO$_{2}$-PPIX components were taken to the improved evaluation of results. Steady state deuterium lamp was used to measure PL emission spectra under \numprint[nm]{339} and \numprint[nm]{405} excitation. PL decay curves were measured using nanosecond pulsed LEDs \numprint[nm]{339}, \numprint[nm]{389} and \numprint[nm]{452}. 


    % \subsection{Production of singlet oxygen}  \label{sec:production_singlet}

    % {\singlet} is generated by irradiation of suspension of nanocomposite material in some solvent in the cuvette. Chemical probe APF was used to monitor generation of {\singlet}. The suspension was about \numprint[mg]{15} of powder, about \numprint[ml]{2,5} of solvent and \numprint[$\mu$l]{150} of APF. Monitoring is based on indirect measurement of luminescence of APF. PL emission spectra for \numprint[nm]{470} excitation was measured. Samples were irradiated by X-ray source on Departement of Nuclear Chemistry, \numprint[kV]{40} and \numprint[mA]{15}.
    
    