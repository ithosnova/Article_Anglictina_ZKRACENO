
\section{Introduction}
\label{s:introduction}

Nanobiotechnology is an emerging branch of~science that deals with the applications of~nanotechnology in the life sciences \cite{jain10}. Nanomaterials are smaller than the human cells and approximately similar in size to biological macromolecules. Nanoparticles attached to~liposomes, polymers, folic acids, antibodies or peptides improve specific passive or~active drug targeting and efficiency of~medical treatments \cite{lammers08}.   
 
Photodynamic Therapy (PDT) is medical treatment technique based on  photosensitized generation of~singlet oxygen. This technique is used to treat various types of~cancer, e.g. skin, lung, esophagus or stomach cancer, and it is an alternative method to chemotherapy and radiotherapy \cite{rollakanti15,takahashi09}. The biggest limitation for PDT is the use of~visible light in therapeutic window of~\SIrange{600}{900}{nm} to irradiate photosensitizers. These wavelengths penetrate the epidermis but cannot reach deeply located organs or tissues. That is why extensive research of~PDT induced by X-rays (PDTX) is ongoing. Addition of~nanoscintillator enables the use of~X-rays instead of~visible light. Nanoscintillators absorb X-rays and convert them to the visible light in absorption range of~photosensitizer \cite{jary13,chen15}. These changes enhance depth penetration and increase the number of~treatable cancer types. \\

In this work I use zinc oxide doped with gallium (ZnO:Ga) as a nanoscintillator. The surface of nanoparticles is modified by silica coating and protoporphyrin IX (PPIX) is used as a photosensitizer. 
