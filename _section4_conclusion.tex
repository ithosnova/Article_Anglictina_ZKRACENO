\newpage

\section{Conclusion}
\label{s:conclusion}


    The XRDP characterization shows that the biofunctionalization does not effect diffraction patterns. Patterns of prepared nanoparticles, silica coated nanoparticles and biofunctionalized nanocomposite are the same. 
    Extensive presence of free PPIX in the sample exhibits broad peak of amorphous phase in the pattern.\\

    Low intensity of luminescence proves that it is important to characterize materials between each step to avoid difficulties during preparation, this is especially true for nanoparticles after reduction heating step. 
    Measurements of PL decay curves show presence of each component in the composite material and it can prove FRET between nanoparticles and photosensitizer. 
    For future research it is advisable to also measure PL decay curves of powder to demonstrate FRET. 
    PL emission spectra and PL decay curves indicate PPIX luminescence under \numprint[nm]{339} excitation that demanding careful evaluation of FRET. 
    Analysis of PL decay curves shows exciton emission from the SiO$_{2}$ shell. The decay time is similar to decay time of PPIX in dilute solution and it is observe at \numprint[nm]{390}.\\

    For future research cuvette holder at FZU enables simple X-ray irradiation of samples and detection of {\singlet}. 
    It is possible to optimize the concentration of APF probe and quencher {\azid} and to use the optimized method for {\singlet} detection using other fluorescence probes or photosensitizers.



\section*{Acknowledgements}
This study has been supported by the Czech Science Foundation under grant GA17-06479S and Student Grant Agency of the Czech Technical University in Prague, project No. SGS14/207/OHK4/3T/14.